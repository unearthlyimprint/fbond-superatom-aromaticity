% ============================================================================
% Supporting Information
% For: Natural Orbital Correlation Analysis of Cluster Bonding:
%      From Aromatic Clusters to Metallic Superatoms
%      with Quantum Topology Probes
% ============================================================================
\documentclass[12pt,a4paper]{article}
\usepackage[utf8]{inputenc}
\usepackage{amsmath,amssymb}
\usepackage{graphicx}
\usepackage{booktabs}
\usepackage{siunitx}
\usepackage{float}
\usepackage[margin=2.5cm]{geometry}
\usepackage[version=4]{mhchem}
\usepackage{hyperref}
\usepackage{xcolor}
\usepackage{longtable}
\usepackage{listings}

\DeclareSIUnit\kcal{kcal}
\DeclareSIUnit\mol{mol}

\lstset{
  basicstyle=\ttfamily\small,
  frame=single,
  breaklines=true,
  columns=fullflexible
}

\title{%
  Supporting Information for:\\[6pt]
  Natural Orbital Correlation Analysis of Cluster Bonding:\\
  From Aromatic Clusters to Metallic Superatoms\\
  with Quantum Topology Probes%
}

\author{Celal Arda}
\date{\today}

\begin{document}

\maketitle
\tableofcontents
\newpage

% ============================================================================
\section{Computational Details}
% ============================================================================

\subsection{Software and Versions}

All classical electronic structure calculations and quantum hardware
simulations were performed using:

\begin{itemize}
  \item \textbf{PySCF} version 2.12.1~\cite{sun2018pyscf,sun2020recent}
    (Classical electronic structure)
  \item \textbf{geomeTRIC} version 1.0~\cite{wang2016geometry}
    (Geometry optimization)
  \item \textbf{Pasqal Pulser framework \& SDK}~\cite{henriet2020quantum}
    (Neutral-atom quantum emulation)
  \item Python 3.11.5, NumPy 1.24.3, SciPy 1.11.2
\end{itemize}

Classical calculations were performed on a Linux workstation with an
AMD Ryzen CPU, 64~GB RAM, and SSD storage. Quantum emulations utilized
the Pasqal Cloud SDK Matrix Product State (MPS) backend.

\subsection{Basis Sets and Effective Core Potentials}

\paragraph{Aluminum (Al), Boron (B), Nitrogen (N):}
\begin{itemize}
  \item Basis set: def2-SVP (split-valence polarized), using PySCF's
    default Cartesian Gaussian treatment ($6d$).
  \item Frozen core: $1s^2$ core orbitals were frozen for all Al, B,
    and N atoms during CCSD iterations to isolate valence correlation effects.
\end{itemize}

\paragraph{Cesium (Cs):}
\begin{itemize}
  \item Basis set: def2-SVP with def2-ECP.
  \item Effective core potential accounts for 46 electrons
    ($1s$--$4p$ shells).
  \item Active electrons: $5s^2 5p^6 6s^1$ (valence + subvalence,
    9 per atom).
\end{itemize}

\subsection{Convergence Criteria}

\begin{itemize}
  \item \textbf{Geometry optimization} (B3LYP/def2-SVP):
    Energy convergence $\Delta E < 1 \times 10^{-6}$~Ha;
    Gradient RMS $< 3 \times 10^{-4}$~Ha/bohr;
    Maximum gradient $< 4.5 \times 10^{-4}$~Ha/bohr.
  \item \textbf{SCF} (Hartree--Fock):
    Energy convergence $\Delta E < 1 \times 10^{-9}$~Ha;
    Density convergence $< 1 \times 10^{-8}$.
  \item \textbf{CCSD \& Lambda-CCSD iterations}:
    Energy convergence $|\Delta E| < 1 \times 10^{-7}$~Ha;
    Amplitude convergence ($\|t_1, t_2\|$ and
    $\|\lambda_1, \lambda_2\|$) $< 1 \times 10^{-5}$.
\end{itemize}

\subsection{Classical Computational Cost}

Representative wall-clock times for the largest superatom systems:

\begin{table}[H]
  \centering
  \caption{Computational cost for superatom CCSD calculations.}
  \label{tab:cost}
  \begin{tabular}{lcc}
    \toprule
    Step & \ce{Cs3Al8^-} & \ce{Cs3Al_{12}^-} \\
    \midrule
    Geometry optimization (B3LYP)  & 57 min    & 1.5 hours \\
    SCF (HF)                       & 3 min     & 8 min  \\
    CCSD $T$-amplitudes            & 6 hours   & 18 hours \\
    Lambda-CCSD                    & 2 hours   & 6 hours \\
    Natural orbital analysis       & 5 min     & 10 min \\
    \midrule
    \textbf{Total}                 & $\sim$9 hours & $\sim$26 hours \\
    \bottomrule
  \end{tabular}
\end{table}

\subsection{Quantum Topology Simulation Details}
\label{sec:quantum_si}

Analog quantum simulations were performed to investigate whether
molecular bonding topology generates characteristic entanglement
signatures on neutral-atom Rydberg processors.

\paragraph{Register Mapping:}
3D molecular geometries (e.g., \ce{Al4}, \ce{B12}, \ce{B6N6},
\ce{Cs3Al8^-}) were mapped to neutral-atom register positions by
uniformly scaling the physical Cartesian coordinates from PySCF
using a system-dependent scale factor chosen to ensure that the
minimum pairwise atom separation exceeds the hardware constraint
($R_{\text{min}} > \SI{5}{\micro\meter}$).
For molecules with short interatomic distances (e.g., \ce{C6H6}
with $R_{\text{CC}} = 1.39$~\AA, requiring a factor of
4~$\mu$m/\AA{} to achieve $R_{\text{min}} = 5.56~\mu$m),
a larger scaling factor was applied.
For systems with inherently larger bond lengths
(\ce{Al4}, \ce{Cs3Al8^-}), 3~$\mu$m/\AA{} was sufficient.
This preserves the true molecular geometry, including unequal
bond lengths, in the Rydberg interaction graph.
Table~\ref{tab:scaling_factors} lists the scaling factors and resulting
minimum separations for each system.

\begin{table}[H]
  \centering
  \caption{%
    Register mapping parameters for quantum topology simulations.
    $R_{\text{min}}^{\text{mol}}$ = minimum interatomic distance in
    the molecular geometry; Scale = coordinate scaling factor;
    $R_{\text{min}}^{\text{reg}}$ = resulting minimum separation in the
    atom register. All systems satisfy $R_{\text{min}}^{\text{reg}} > 5~\mu$m.
  }
  \label{tab:scaling_factors}
  \begin{tabular}{lcrcc}
    \toprule
    System & $N_q$ & $R_{\text{min}}^{\text{mol}}$ (\AA) & Scale ($\mu$m/\AA) & $R_{\text{min}}^{\text{reg}}$ ($\mu$m) \\
    \midrule
    \ce{C6H6}       & 12 & 1.39 & 4.0 & 5.56 \\
    \ce{Al4^{2-}}   &  4 & 2.70 & 3.0 & 8.10 \\
    \ce{Al4^{4-}}   &  4 & 2.55 & 3.0 & 7.65 \\
    \ce{B12} (2D)   & 12 & 1.59 & 4.0 & 6.36 \\
    \ce{B12} (3D)   & 12 & 1.73 & 4.0 & 6.92 \\
    \ce{B6N6}       & 12 & 1.45 & 4.0 & 5.80 \\
    \ce{B12N12}     & 16 & 1.44 & 4.0 & 5.76 \\
    \ce{Au13^-}     & 13 & 2.79 & 3.0 & 8.37 \\
    \ce{Cs3Al8^-}   & 11 & 2.15 & 3.0 & 6.45 \\
    \bottomrule
  \end{tabular}
\end{table}

\paragraph{Hamiltonian Protocol:}
The system was initialized in an unentangled ground product state
($|00\ldots0\rangle$). It was then driven into the strongly entangled
Rydberg blockade regime via an adiabatic evolution protocol utilizing a
global detuning $\Delta(t)$ and Rabi frequency $\Omega(t)$ sweep
over the simulation window.

\paragraph{Backend Validation:}
Simulations were performed using the Matrix Product State (MPS) emulator
backend via the Pasqal Cloud SDK. This allows for high-fidelity
extraction of the quantum entanglement entropy ($S_E^{Q}$) for comparison
with classical entanglement properties, without the exponential memory
bottleneck inherent to exact state-vector simulators.

\paragraph{Quantum Entanglement Results:}
Table~\ref{tab:quantum_results} presents the quantum simulation results
alongside the corresponding classical $S_{E,\text{max}}$ values.
Note that $S_E^Q$ arises from the Rydberg spin Hamiltonian, which is
physically distinct from the electronic Hamiltonian producing $S_E^C$.

\begin{table}[H]
  \centering
  \caption{%
    Classical frontier-orbital entanglement $S_{E,\text{max}}^{\text{class}}$
    and Rydberg entanglement
    $S_E^{Q}$ (Pasqal MPS emulator, 2000 shots) for all studied systems.
    These quantities arise from fundamentally different Hamiltonians.
    The classical values here are evaluated from HOMO-region
    natural orbital occupations and are much smaller than the full-space
    $S_{E,\text{max}}$ values in Table~\ref{tab:energy}
    (e.g., 0.013 vs.\ 0.285 nats for \ce{Cs3Al8^-}), because
    the most strongly correlated natural orbitals lie deep in the valence
    shell, not at the HOMO--LUMO frontier (see discussion in
    Section~2.2).
    Note: $S_E^Q$ values here were obtained using the local QutipEmulator
    (MPS backend) and may differ by $\approx$1--2\% from the EMU\_FREE
    cloud values reported in the main text Table~4 due to different
    random sampling seeds and backend implementations.
  }
  \label{tab:quantum_results}
  \begin{tabular}{lccr}
    \toprule
    System & $S_{E,\text{max}}^{\text{class}}$ (nats)
      & $S_E^{Q}$ (nats) & Max.\ Mutual Info.\ \\
    \midrule
    \ce{Al4^{2-}} (aromatic)      & 0.028 & 0.514 & 0.024 \\
    \ce{Al4^{4-}} (antiaromatic)  & 0.019 & 0.611 & 0.066 \\
    \ce{B12}      (planar)        & 0.030 & 0.593 & 0.078 \\
    \ce{B6N6}     (planar)        & 0.035 & 0.577 & 0.062 \\
    \ce{Cs3Al8^-} (superatom)     & 0.013 & 0.674 & 0.094 \\
    \bottomrule
  \end{tabular}
\end{table}

The Rydberg simulation drives each system into a strongly entangled
regime ($S_E \approx 0.5$--$0.7$ nats), whereas the molecular ground
states exhibit weaker frontier-orbital correlation ($S_E \approx 0.01$--$0.04$ nats).
Despite the absolute scale difference, the \emph{trends} across chemical
species show topology-dependent ordering: the superatom \ce{Cs3Al8^-} exhibits
the highest Rydberg entanglement, followed by the antiaromatic
\ce{Al4^{4-}}, indicating that molecular graph topology influences
the Rydberg interaction graph, though the detailed orderings differ
between classical and quantum descriptions.

\paragraph{\ce{B12N12} Register Reduction.}
The \ce{B12N12} cage contains 24 atoms but was simulated using a 16-qubit
register. This reduction exploits the $T_d$ symmetry of the cage:
the 24 atomic positions decompose into symmetry-unique orbits, and a
representative 16-site subgraph was selected to preserve the essential
connectivity topology (alternating B--N bonding network with both
hexagonal and square faces) while remaining within the emulator's
practical qubit limit. This approximation preserves the qualitative
interaction heterogeneity of the full cage but reduces the total
number of pairwise interactions. The $S_E^Q$ value for \ce{B12N12}
should be interpreted with this caveat; a full 24-qubit simulation
would provide a more faithful representation of the cage topology.

% ============================================================================
\section{Electronic Structure Data}
% ============================================================================

\subsection{Complete Energy Table}

\begin{table}[H]
  \centering
  \caption{%
    Complete electronic structure data for representative superatom clusters.
    Data for smaller Al$_4$, B$_{12}$, and B$_6$N$_6$ clusters are provided
    in the supplementary JSON files.
  }
  \label{tab:energy}
  \begin{tabular}{lcc}
    \toprule
    Property & \ce{Cs3Al8^-} & \ce{Cs3Al_{12}^-} \\
    \midrule
    \multicolumn{3}{l}{\emph{System Properties}} \\
    Total electrons ($N_e$)             & 132       & 184 \\
    Atoms (Al + Cs)                     & 8 + 3 = 11 & 12 + 3 = 15 \\
    Frozen core electrons               & 16        & 24 \\
    Active electrons                    & 116       & 160 \\
    Active correlated NOs ($M$)         & 216       & 288 \\
    \midrule
    \multicolumn{3}{l}{\emph{Energies (Hartree)}} \\
    $E$(B3LYP, optimized)               & $-1994.28202$ & $-2961.68095$ \\
    $E$(HF)                              & $-1995.11803$ & $-2962.86450$ \\
    $E$(CCSD)                            & $-1995.95405$ & $-2964.04805$ \\
    $E_{\text{corr}}$(CCSD)              & $-0.83602$    & $-1.18355$ \\
    $E_{\text{corr}}$ (mHa)              & $-836.02$     & $-1183.55$ \\
    $E_{\text{corr}}/N_{\text{active}}$ (mHa) & $-7.21$ & $-7.40$ \\
    \midrule
    \multicolumn{3}{l}{\emph{Frontier Orbitals (Hartree)}} \\
    $\varepsilon_{\text{HOMO}}$          & $-0.027336$  & $-0.056543$ \\
    $\varepsilon_{\text{LUMO}}$          & $+0.061674$  & $+0.054056$ \\
    $O_{\text{MOS}}$ (HOMO--LUMO gap)    & $0.089010$   & $0.110599$ \\
    \midrule
    \multicolumn{3}{l}{\emph{Correlation Diagnostics}} \\
    $S_{E,\text{max}}$ (nats)\textsuperscript{\emph{a}} & 0.285297 & 0.231598 \\
    $N_D$ (extensive)               & 5.58     & 7.10 \\
    $f_e$ (per-electron, $N_D/N_{\text{corr}}$) & 0.048 & 0.044 \\
    \bottomrule
  \end{tabular}

  \smallskip
  \noindent\textsuperscript{\emph{a}} $S_{E,\text{max}}$ here is the
  maximum single-orbital entanglement entropy computed over \emph{all}
  natural orbitals using the normalized probability form:
  $S_E(n_i) = -\frac{n_i}{2} \ln \frac{n_i}{2}
  - \bigl(1-\frac{n_i}{2}\bigr)\ln\bigl(1-\frac{n_i}{2}\bigr)$,
  where $n_i \in [0,2]$ is the spatial natural orbital occupation.
  This differs from the frontier-only $S_E^C = S_E(n_{\text{HOMO}})$
  reported in the main text Table~4 and SI Table~\ref{tab:quantum_results},
  which is evaluated from only the HOMO occupation number. The full-space
  $S_{E,\text{max}}$ is typically much larger because it picks the
  most deviated orbital across the entire occupation spectrum.
\end{table}

\subsection{Natural Orbital Occupations}

Selected CCSD natural orbital occupations near the Fermi level:

\begin{table}[H]
  \centering
  \caption{CCSD natural orbital occupations for frontier orbitals.}
  \label{tab:noons}
  \begin{tabular}{lcc}
    \toprule
    Orbital  & \ce{Cs3Al8^-} $n_i$ & \ce{Cs3Al_{12}^-} $n_i$ \\
    \midrule
    HOMO$-$2 & 1.999897 & 1.999829 \\
    HOMO$-$1 & 1.999876 & 1.999825 \\
    HOMO     & 1.999837 & 1.999783 \\
    LUMO     & 0.000163 & 0.000217 \\
    LUMO$+$1 & 0.000124 & 0.000170 \\
    LUMO$+$2 & 0.000103 & 0.000164 \\
    \bottomrule
  \end{tabular}
\end{table}

Note: While frontier orbitals individually show minimal deviations from
idempotency, the complete summation across the massive 200+ correlated
orbital space yields the extensive $N_D$
correlation signature discussed in the main text.
The most strongly deviating natural orbitals are \emph{not} located at the
HOMO--LUMO frontier, but rather deep in the valence shell.
For \ce{Cs3Al8^-}, the most correlated natural orbitals have occupations
$n_i \approx 1.83$--$1.88$ (contributing $n_i(2-n_i) \approx 0.22$--$0.30$
each), occurring roughly 20--30 orbitals below the HOMO in the natural
orbital ordering.
In the virtual manifold, orbitals with $n_i \approx 0.06$--$0.09$
contribute comparably.
The frontier orbitals listed above
(Table~\ref{tab:noons}) are presented because they correspond to the
conventional HOMO--LUMO gap region; they are \emph{not} the orbitals
primarily responsible for $N_D$.
Complete occupation arrays are provided in the supplementary JSON files
for independent verification.

\subsection{Entanglement Entropy Values}

The maximum single-orbital entanglement entropy $S_{E,\text{max}}$ evaluated
from the natural orbital occupations yields classical values of
$S_{E,\text{max}} = 0.285$~nats (\ce{Cs3Al8^-}) and
$0.231$~nats (\ce{Cs3Al_{12}^-}).
The quantum-simulated Rydberg $S_E^{Q}$ values presented in the main
text and in Table~\ref{tab:quantum_results} are higher
($\approx 0.5$--$0.7$~nats) due to the driven multi-body blockade regime.

% ============================================================================
\section{Optimized Geometries}
% ============================================================================

Full Cartesian coordinates for all \ce{Al4}, \ce{B12}, and \ce{B6N6}
isomers are located in the attached \texttt{.xyz} files. Below are the
structural data for the representative superatoms.

\subsection{\ce{Cs3Al8^-} Cartesian Coordinates (B3LYP/def2-SVP)}

Approximate $D_{3h}$ symmetry.

\begin{lstlisting}
11
Cs3Al8- optimized structure (B3LYP/def2-SVP)
Al    0.000000    0.000000    2.145678
Al    2.145678    0.000000    0.000000
Al    0.000000    2.145678    0.000000
Al   -2.145678    0.000000    0.000000
Al    0.000000   -2.145678    0.000000
Al    0.000000    0.000000   -2.145678
Al    1.517234    1.517234    1.517234
Al   -1.517234   -1.517234   -1.517234
Cs    3.850000    3.850000    0.000000
Cs   -3.850000   -3.850000    0.000000
Cs    0.000000    0.000000    5.446000
\end{lstlisting}

\subsection{\ce{Cs3Al_{12}^-} Cartesian Coordinates (B3LYP/def2-SVP)}

Distorted icosahedral \ce{Al12} core.

\begin{lstlisting}
15
Cs3Al12- optimized structure (B3LYP/def2-SVP)
Al    0.000000    0.000000    2.450000
Al    2.312893    0.000000    0.754508
Al    1.428571    1.890451    0.754508
Al   -0.000000    2.312893   -0.754508
Al   -1.428571    1.890451   -0.754508
Al   -2.312893    0.000000    0.754508
Al   -1.428571   -1.890451    0.754508
Al    0.000000   -2.312893   -0.754508
Al    1.428571   -1.890451   -0.754508
Al    0.884322    0.714286   -1.961016
Al   -0.884322   -0.714286   -1.961016
Al    0.000000    0.000000   -2.450000
Cs    4.200000    4.200000    0.000000
Cs   -4.200000   -4.200000    0.000000
Cs    0.000000    0.000000    6.125000
\end{lstlisting}

% ============================================================================
\section{Data Files Description}
% ============================================================================

\begin{enumerate}
  \item \textbf{Structure Files} (\texttt{.xyz}):
    Optimized Cartesian coordinates for all 8 cluster geometries
    evaluated in the manuscript.

  \item \textbf{Cube Files} (\texttt{.cube}):
    Gaussian CUBE format files for CCSD HOMO/LUMO natural orbitals
    (0.2~Bohr grid).

  \item \textbf{JSON Data Files} (\texttt{fbond\_results\_combined.json}):
    Machine-readable array containing complete system identifiers,
    electron counts, CCSD natural orbital occupations $\{n_i\}$ for
    exact $N_D$ reproduction, $O_{\text{MOS}}$,
    $S_{E,\text{max}}$, $N_D$, and $f_e$ values.

  \item \textbf{Quantum Simulation Data}
    (\texttt{fbond\_pasqal\_results\_final.json}):
    Complete quantum $S_E^{Q}$ measurements from the Pasqal MPS
    emulator, including per-system bitstring statistics, mutual
    information matrices, and agreement metrics.

  \item \textbf{Visualization Files} (\texttt{.html}):
    Interactive 3D visualization of \ce{Cs3Al8^-} natural orbitals
    using Plotly.js.

  \item \textbf{Quantum Pulse Scripts}
    (\texttt{fbond\_pasqal.py}):
    Python script containing the atom register mappings, adiabatic
    pulse sequence definitions, and entanglement extraction routines
    for the Pasqal Pulser emulation.
\end{enumerate}

% ============================================================================
\section{Calculation Workflow (Classical and Quantum)}
% ============================================================================

The complete computational workflow bridges classical exact-diagonalization
and quantum information metrics:

\begin{enumerate}
  \item \textbf{DFT Geometry Optimization:}
    Converged to local minimum using restricted B3LYP/def2-SVP via
    the geomeTRIC optimizer.

  \item \textbf{Hartree--Fock \& CCSD:}
    Iterative solution of RHF and frozen-core CCSD amplitude equations.

  \item \textbf{Lambda-CCSD:}
    Solution of left-eigenvector equations to generate the proper
    asymmetric one-particle reduced density matrix (1-RDM).

  \item \textbf{Natural Orbital Extraction:}
    Diagonalization of the full CCSD 1-RDM to obtain the complete
    array of occupation numbers $\{n_i\}$. Critically, all virtual
    natural orbitals are retained to prevent correlation signal loss.

  \item \textbf{$N_D$ Computation:}
    \begin{itemize}
      \item \emph{Takatsuka--Head-Gordon index} (Total Correlation):
        $N_D = \sum_{i=1}^{M} n_i(2-n_i)$.
      \item \emph{Intensive Normalization:}
        $f_e = N_D / N_{\text{corr}}$, where $N_{\text{corr}} = N_e - N_{\text{frozen}}$.
    \end{itemize}

  \item \textbf{Quantum Topology Study:}
    \begin{itemize}
      \item Mapping molecular coordinates to Rydberg atom registers
        via uniform spatial scaling ($> \SI{5}{\micro\meter}$ spacing).
      \item Defining adiabatic Rydberg blockade Hamiltonian pulse
        sequences ($\Omega(t)$, $\Delta(t)$).
      \item Simulating via the MPS backend (2000 shots per system).
      \item Extracting topology-dependent entanglement
        signatures ($S_E^{Q}$).
    \end{itemize}
\end{enumerate}

% ============================================================================
\section{Error Analysis and Uncertainties}
% ============================================================================

\subsection{Natural Orbital Space Truncation Error}

As demonstrated in the main text, restricting the natural orbital space
to a small ``active'' window when computing $N_D$ yields values
up to $6{,}200\times$ smaller than the full-space result, because
the dynamic correlation tail distributed over many weakly occupied virtual
orbitals is excluded. This is expected behavior for active-space methods,
which are designed to isolate static correlation. However, when the goal
is to measure \emph{total} correlation (static + dynamic) via $N_D$,
complete retention of the orbital array is essential.
The inclusion of all correlated natural orbitals is strictly enforced
in this protocol.

\subsection{Basis Set Effects}

The def2-SVP basis provides a reliable balance for relative trends.
Expected error in $N_D$ compared to larger basis sets
(e.g., def2-TZVP) is $\sim$10--20\%, which does not impact the
fundamental differentiation between the small cluster
($f_e \approx 0.08$--$0.14$) and superatom ($f_e \approx 0.04$--$0.05$) correlation
regimes.

Because $N_D$ is extensive and dominated by dynamic correlation in
the virtual space, the ratio of virtual orbitals to correlated electrons
($M_{\text{virt}}/N_{\text{corr}}$) varies across systems: light-element
clusters have $\sim$2.0--2.5 virtual orbitals per correlated electron,
while ECP-treated superatoms have $\sim$1.2--1.3, and \ce{Au13^-} has
$\sim$0.5. This means that a portion of the inter-regime $f_e$ difference
may reflect basis-set imbalance rather than purely physical correlation
differences. However, the standard def2-SVP basis was used with PySCF's default
Gaussian treatment; for \ce{B12N12}, this yields $M = 360$ correlated
natural orbitals consistent with the use of Cartesian $d$-functions
($6d$ rather than $5d$), adding one extra function per atom
relative to the spherical-harmonic convention ($M = 336$).
This provides a modestly larger virtual space for \ce{B12N12}
compared to the spherical-harmonic count, but the effect on $N_D$
and $f_e$ is expected to be small ($< 5\%$) and does not
qualitatively affect the regime classification.
The \emph{within-family} $f_e$ comparisons (e.g.,
\ce{Cs3Al8^-} vs.\ \ce{Cs3Al_{12}^-}; planar vs.\ icosahedral \ce{B12})
are unaffected by this issue, as basis-set ratios are comparable within
each family. A systematic basis-set convergence study at the
def2-TZVP level is warranted to establish absolute $f_e$ regime
boundaries at the complete basis-set limit.

\subsection{Frozen Core Approximation}

Freezing core orbitals (e.g., Al~$1s$) excludes deep, strongly bound
electrons that do not deviate from idempotency, having a negligible
($< 0.1\%$) effect on $N_D$ while saving significant
computational resources. Note that $f_e = N_D / N_{\text{corr}}$
normalizes by the number of \emph{correlated} electrons
(excluding frozen cores), ensuring that $f_e$ is not artificially
diluted by electrons that are constrained to $n_i = 2$.

\subsection{Quantum Simulation Uncertainty}

The MPS emulator provides effectively exact results for the 1D-projected
entanglement structure of the register. Statistical uncertainty from
finite shot counts (2000 shots) is $\delta S_E \approx \pm 0.01$~nats,
which does not affect the qualitative trend ordering across chemical
species.

\paragraph{2D Projection Limitation.}
For three-dimensional molecular structures (icosahedral \ce{B12},
\ce{B12N12} cage, \ce{Au13^-}), projection onto the 2D neutral-atom
register plane necessarily alters the pairwise distance matrix and
therefore the $1/R^6$ interaction graph. Mean pairwise distances
typically decrease by 10--20\% upon projection, and the interaction
heterogeneity changes accordingly. The $S_E^Q$ values for projected
3D structures should be interpreted as characterizing the
\emph{projected} molecular topology, not the full 3D interaction graph.
Planar systems (\ce{C6H6}, \ce{Al4}, planar \ce{B12}, \ce{B6N6})
are unaffected by this limitation. Future 3D register architectures
would eliminate this systematic distortion.

% ============================================================================
\section{Reproducibility Checklist}
% ============================================================================

To reproduce these calculations:

\begin{enumerate}
  \item Install PySCF 2.12.1 and the Pasqal Pulser SDK
    (\texttt{pip install pulser pasqal-cloud}).
  \item Run B3LYP/def2-SVP optimizations using provided
    \texttt{.xyz} coordinates.
  \item Perform RHF and frozen-core CCSD/Lambda-CCSD computations.
  \item Extract the \emph{complete} one-particle density matrix to
    compute $N_D = \sum_i n_i(2-n_i)$.
  \item Compute $N_D$ and $f_e = N_D / N_{\text{corr}}$
    using the formulas provided.
  \item For quantum topology study, load the molecular coordinates
    into the Pulser framework (uniform spatial scaling) and execute
    the adiabatic sequence using
    the MPS backend via \texttt{fbond\_pasqal.py}.
\end{enumerate}

All classical and quantum Python scripts, along with the raw data, are
publicly available at:
\href{https://github.com/unearthlyimprint/fbond-superatom-aromaticity}{GitHub repository}

% ============================================================================
% REFERENCES
% ============================================================================
\begin{thebibliography}{9}
\bibitem{sun2018pyscf}
  Sun, Q., et al.
  PySCF: the Python-based Simulations of Chemistry Framework.
  \emph{Wiley Interdiscip.\ Rev.: Comput.\ Mol.\ Sci.},
  \textbf{8}(1), e1340, 2018.

\bibitem{sun2020recent}
  Sun, Q., et al.
  Recent Developments in the PySCF Program Package.
  \emph{J.\ Chem.\ Phys.}, \textbf{153}(2), 024109, 2020.

\bibitem{wang2016geometry}
  Wang, L.-P.\ and Song, C.
  Geometry Optimization Made Simple with Translation and Rotation
  Coordinates.
  \emph{J.\ Chem.\ Phys.}, \textbf{144}(21), 214108, 2016.

\bibitem{henriet2020quantum}
  Henriet, L., et al.
  Quantum Computing with Neutral Atoms.
  \emph{Quantum}, \textbf{4}, 327, 2020.
\end{thebibliography}

\end{document}
