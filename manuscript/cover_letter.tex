\documentclass[12pt,a4paper]{letter}
\usepackage[utf8]{inputenc}
\usepackage[margin=2.5cm]{geometry}
\usepackage{hyperref}
\usepackage{xcolor}

\signature{Celal Arda\\Independent Researcher\\celal.arda@outlook.de}
\address{}

\begin{document}

\begin{letter}{%
  Editorial Office\\
  ACS Omega\\
  American Chemical Society
}

\opening{Dear Editor,}

Please find enclosed our manuscript entitled \textbf{``Natural Orbital
Correlation Analysis of Cluster Bonding: From Aromatic Clusters to
Metallic Superatoms with Quantum Topology Probes''} for consideration
as a Research Article in \textit{ACS Omega} (resubmission of
ao-2026-018388).

\medskip\noindent\textbf{Summary and Significance.}\quad
We present a systematic study of electron correlation across 11
molecular and cluster systems (42--360 correlated natural orbitals)
using the Takatsuka--Head-Gordon index $N_D = \sum_i n_i(2-n_i)$
computed from CCSD natural orbital occupations over the
\emph{complete} correlated orbital space.

Our most immediately impactful methodological finding is that
active-space truncation---a common practice in the multireference
diagnostics community---can underestimate $N_D$ by factors exceeding
\textbf{6,200$\times$} (Table~3). This demonstrates that meaningful
NOON-based correlation diagnostics require the full natural orbital
array, not just a handful of ``active'' orbitals. This finding has
broad implications for how the community interprets $T_1$ diagnostics,
fractional occupation numbers, and related multireference indicators.

We introduce the per-electron correlation density
$f_e = N_D/N_{\mathrm{corr}}$ as a complementary intensive measure,
revealing two distinct bonding regimes: small clusters exhibit
``narrow but deep'' correlation ($f_e \approx 0.08$--$0.14$), while
metallic superatoms display ``wide but shallow'' correlation
($f_e \approx 0.04$) spread across 96\% of correlated orbitals.

In a complementary study, we embed the molecular bonding topologies as
interaction graphs on a Pasqal neutral-atom quantum processor and show
that the resulting Rydberg entanglement signatures are not determined
by simple graph-theoretic metrics (Spearman $p > 0.05$ for all tested
descriptors), but rather reflect the specific chemical bonding
character of each system. This establishes that molecular topology
encodes information accessible to analog quantum hardware beyond what
structural descriptors alone capture.

\medskip\noindent\textbf{Scope and Readership.}\quad
The manuscript bridges computational chemistry (CCSD correlation
analysis), cluster science (Al, B, Au, Cs--Al superatoms), and quantum
computing (analog Rydberg simulations on Pasqal hardware). We believe
this interdisciplinary scope is well suited to the broad readership of
\textit{ACS Omega}.

\medskip\noindent\textbf{Manuscript Details.}\quad
\begin{itemize}
  \item \textbf{Type:} Research Article
  \item \textbf{Figures:} 7 (all original)
  \item \textbf{Tables:} 5 (main text) + 3 (Supporting Information)
  \item \textbf{Supporting Information:} Computational details, complete
    energy tables, natural orbital occupation arrays, Pulser simulation
    scripts, and reproducibility information
  \item \textbf{Conflicts of interest:} None
  \item \textbf{Prior publication:} A preliminary version was deposited
    on ChemRxiv (DOI: 10.26434/chemrxiv-2025-XXXXX)
\end{itemize}

\medskip\noindent
We confirm that this work is original, has not been published
elsewhere, and is not under consideration by any other journal. All
data and simulation scripts are available in the Supporting
Information.

Thank you for considering our manuscript.

\closing{Sincerely,}

\end{letter}
\end{document}
